\myChapter{Dependability e Security}
I veicoli a guida autonoma appartengono alla categoria dei sistemi critici. Un sistema critico è un sistema il cui malfunzionamento può provocare danni
considerati inacettabili. Questi includono danni a oggetti di valore, danni ambientali e nei casi piu gravi, il ferimento o addirittura la morte delle persone.
Per garantire che un tale sistema operi nel modo più sicuro possibile è necessario analizzare tutti i fattori che possono portare a  un fallimento irreversibile.

La \textbf{dependability} è la capacità di un sistema di fornire un servizio sul quale è possibile fare affidamento in modo giustificato \cite{tax}.
Essa viene suddivisa in 3 categorie:
\begin{itemize}
    \item Attributi,
    \item Minacce,
    \item Mezzi di Raggiungimento.
\end{itemize}
\begin{figure}[h]
    \includegraphics[width = \linewidth]{dep.png}
    \caption{tassonomia  della dependability\cite{dep}}
    \label{fig:dep}
\end{figure}
\section{Attributi}
La dependability comprende i seguenti attributi:
\begin{itemize}
    \item \textbf{availability}: disponibilità del servizio corretto,
    \item \textbf{reliability}: stabilità del servizio corretto,
    \item \textbf{safety}: assenza di conseguenze catastrofiche sull'utente e sull'ambiente,
    \item \textbf{integrity}: assenza di alterazioni improprie al sistema,
    \item \textbf{maintainability}: capacità di subire modifiche e riparazioni.
\end{itemize}
Quando si considera la \textbf{security} è necessario specificare un ulteriore attributo: la confidentiality, ovvero la mancata divulgazione di informazioni non 
autorizzata. La security è composta da confidentiality, integrity e availability.
\section{Mezzi di raggiungimento}
Nel corso degli anni si sono sviluppate molte metodologie per raggiungere dependability e security. Tali metodologie possono essere raggruppate in quattro macrocategorie:
\begin{itemize}
    \item \textbf{fault prevention}: mezzi per prevenire l'occorrenza e l'introduzione di guasti;
    \item \textbf{fault tolerance}: mezzi per evitare fallimenti di servizio quando sono presenti dei guasti;
    \item \textbf{fault removal}: mezzi per ridurre numero e gravità dei guasti;
    \item \textbf{fault forecasting}: mezzi per stimare numero, incidenza futura e probabili conseguenze di guasti.
\end{itemize}
Fault prevention e fault tolerance portano al conseguimento della dependability mentre fault removal e fault forecasting sono i mezzi di validazione
\section{Minacce}
Le maggiori minacce per la dependability sono i \textbf{guasti}(faults in inglese). I guasti hanno varie cause e causano gli \textbf{errori}. Un errore può causarne altri fino a propagarsi
al di fuori dei confini del sistema. Quando ciò avviene, si verifica un \textbf{fallimento}. Un fallimento è la situazione in cui il servizio fornito
è diverso dal servizio corretto.
\begin{figure}[h]
    \includegraphics[width=\linewidth]{chain.png}
    \caption{Catena guasto-errore-fallimento}
\end{figure}
\subsection{Tipi di guasti}
I guasti vengono classificati secondo 8 parametri, i quali creano le classi di guasto elementari, mostrate in Figura \ref{fig:tax}.
\begin{figure}[h]
    \includegraphics[width = \linewidth]{faults.png}
    \caption{tassonomia dei guasti\cite{tax}}
    \label{fig:tax}
\end{figure}
 I diversi tipi di guasto sono raggruppati in tre categorie:
 \begin{itemize}
     \item \textbf{Development faults}: i guasti che avvengono in fase di sviluppo;
     \item \textbf{Physical faults}: i guasti che riguardano l'hardware;
     \item \textbf{Interaction faults}: tutti i guasti causati dall'interazione con l'ambiente esterno.
 \end{itemize}
 \newpage
 \subsubsection{Guasti causati dall'azione umana}
 I guasti sul quale ci concentriamo sono quelli di origine umana . Questi guasti possono essere causati da una mancanza di 
 azioni ove necessarie(omission faults), oppure da azioni che si rilevano essere sbagliate(commision faults). Vengono
 suddivisi in due tipi, sulla base dell'\emph{obiettivo} dello sviluppatore o dell'essere umano che lo causa:
 \begin{itemize}
     \item guasti involontari, causati  accidentalmente senza lo scopo di arrecare danno;
     \item guasti maligni, causati in modo volontario per arrecare danni al sistema durante l'uso.
 \end{itemize}
 I guasti maligni hanno  come scopo la negazione del servizio(denial of service), l'accesso a informazioni confidenziali
 o la modifica impropria di un sistema.
 Vengono raggruppati in due classi:
 \begin{itemize}
     \item \textbf{guasti a livello logico}: comprendono virus, worms, bombe logiche ecc\dots;
     \item \textbf{tentativi di intrusione}, anche usando mezzi fisici.
 \end{itemize}
 In entrambi i casi si sfrutta una vulnerabilità di un sistema per ottenerne il pieno controllo(exploit).
 La vulnerabilità solitamente è a livello software ed è causata dall'azione involontaria degli sviluppatori.

