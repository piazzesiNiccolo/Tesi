\newpage
\section{Adversarial Robustness Toolbox}
L'Adversarial Robustness Toolbox (ART) è una libreria Python che aiuta sviluppatori e ricercatori nella difesa dei modelli di machine learning dagli adversarial examples, rendendoli più sicuri 
ed affidabili. Creare tali difese significa certificare e verificare la robustezza con metodi come il preprocessing degli input,  l'aumento del training set
con esempi adversarial e fare uso di metodologie di rilevamento di input modificati. L'ART fornisce degli attacchi grazie ai quali valutare la sicurezza dei modelli
testati. Nella libreria sono presenti interfacce standard per le librerie di machine learning più popolari. L'architettura dell'ART rende semplice combinare i vari approcci  ed è progettato sia per i ricercatori che vogliono eseguire esperimenti su larga scala di benchmarking di attacchi e difese, sia per gli sviluppatori 
che vogliono rilasciare applicazioni che fanno uso di machine learning in modo sicuro \cite{art2018}.
\subsection{La struttura della libreria}
L'ART è diviso in vari sottomoduli:
\begin{itemize}
    \item art.attacks
    \item art.classifiers
    \item art.defences
    \item art.detection
    \item art.metrics
    \item art.poison\_detection
\end{itemize}

\subsubsection{art.attacks}
Modulo nel quale sono definite  le classi che implementano gli attacchi forniti. Ogni classe eredita dalla classe base  $Attack$ il metodo $generate$: Questo metodo ha come parametro l'array x contenente l'input originale e restituisce
un array contenente l'input perturbato. L'implementazione concreta di $generate$ varia a seconda dell'attacco. Gli attacchi sono divisi in tre categorie: \textbf{Evasion Attacks},
\textbf{Poisoning Attacks} e \textbf{Extraction Attacks}. 
\paragraph{Evasion Attacks}
Si applica una modifica impercettibile all'input per causare la misclassificazione  del modello attaccato. Si distinguono in attacchi whitebox  e attacchi blackbox a seconda che la conoscenza dei pesi del modello
sia necessaria o meno. Gli attachi blackbox infatti, hanno bisogno della sola previsione finale. Al momento della scrittura, gli evasion attacks presenti nella libreria sono:
 Threshold Attack, Pixel Attack, HopSkipJump Attack, High Confidence Low Uncertainty adversarial examples, Projected Gradient Descent, NewtonFool, Elastic net attack,
Spatial transformation attack, Query-efficient black-box attack, Zeroth-order optimization attack, Decision-based attack / Boundary attack, Adversarial patch, Decision tree attack,
Carlini \& Wagner(C\&W) L\_2 and L\_inf attacks, Basic iterative method, Jacobian saliency map, Universal perturbation, DeepFool, Virtual adversarial method, Fast gradient method

\paragraph{Poisoning Attacks}
Si modifica il training set per ridurre l'accuratezza del processo di learning. Il poisoning include la modifica degli esempi nel dataset, l'iniezione di dati "maligni" o la modifica
delle etichette. Al momento della scrittura, i poisoning attacks presenti nella libreria sono: Poisoning Attack on SVM, Backdoor Attack

\paragraph{Extraction Attacks} Si cerca di sviluppare un modello sulla base di un modello proprietario e chiuso, "rubando" il comportamento del modello attaccato. Al momento della scrittura, gli extraction attacks presenti nella libreria sono: 
Functionally Equivalent Extraction, Copycat Confidence, KnockoffNets

\subsubsection{art.classifers}
Modulo nel quale sono definite le classi che permettono di usare l'ART insieme a modelli sviluppati con librerie di Machine Learning esterne.  Le liberie attualmente supportate sono: 
Tensorflow, Keras, PyTorch, MXNet, Scikit-learn, XGBoost,LightGBM,CatBoost e Gpy. Le classi svolgono il ruolo di "contenitore", fornendo metodi per addestrare e testare i modelli senza dover accedere direttamente
ad essi.
\subsubsection{art.defences}
In questo modulo vengono definite le classi che implementano vari tipi di difese contro gli adversarial attacks. Le difese sono divise in:difese preprocessing, difese postprocessing
e difese basate su Transformer.

\paragraph{Difese preprocessing}
Difese che agiscono applicando modifiche agli input prima che vengano classificati.
     \begin{itemize}
        \item Thermometer encoding
        \item total variance minimization
        \item PixelDefend
        \item Gaussian data augmentation
        \item Feature squeezing
        \item Spatial smoothing
        \item JPEG compression
        \item Label smoothing
        \item Virtual adversarial training
    \end{itemize}

\paragraph{Difese postprocessing}
Difese che agiscono modificando l'output generato dal modello.
\begin{itemize}
        \item Reverse Sigmoid
        \item Random Noise 
        \item Class labels
        \item High Confidence
        \item Rounding
    \end{itemize}
\paragraph{difese basate sul training}
Difese il cui è obiettivo è rendere più robusto un modello durante la fase di addestramento
 \begin{itemize}
        \item Adversarial training
        \item Adversarial training Madry PGD
\end{itemize}
\paragraph{Difese basate su Transformer}
Difese che irrobustiscono i modelli a cui sono applicate utilizzando tecniche basate sul modello Transformer.
    \begin{itemize}
        \item Defensive Distillation
    \end{itemize}

\subsubsection{art.detection e art.poison\_detection}
Modulo nel quale vengono definite le classi che permettono la rilevazione di adversarial inputs e  di dataset avvelenati.
\begin{itemize}
    \item adversarial detection \begin{itemize}
        \item rilevatore base degli input
        \item rilevatore addestrato su uno specifico livello
        \item rilevatore basato su Fast Generalized Subset Scan
    \end{itemize}
    \item poisoning detection \begin{itemize}
        \item rilevazione basata sull'analisi dell'attivazione
        \item rilevazione basata sulla provenienza dei dati
    \end{itemize}
\end{itemize}
\subsubsection{art.metrics}
In questo modulo sono definite alcune metriche per verificare, validare e certificare sicurezza e robustezza dei modelli in analisi. 

\begin{itemize}
    \item Clique Method Robustness Verification
    \item Randomized smoothing
    \item CLEVER
    \item Loss sensitivity
    \item Empirical robustness
\end{itemize}