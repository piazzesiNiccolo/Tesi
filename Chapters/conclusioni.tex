\myChapter{Conclusioni e sviluppi}
In questa Tesi abbiamo studiato l'iniezione di adversarial attacks su modelli di guida autonoma basati su reti neurali convoluzionali (CNN) \cite{art2018}. In particolare
abbiamo utilizzato la libreria python Adversarial Robustness Toolbox (ART). Questa libreria fornisce vari attacchi per studiare la robustezza di un
modello decisionale. Abbiamo studiato gli attacchi presenti nell'ART  e individuato quali di essi fosse interessante applicare a un modello.\\

Il  modello  scelto per l'iniezione è stato LearningByCheating(LBC). LBC è un modello di guida autonoma che gira sul simulatore 
CARLA. La sperimentazione si è divisa in due fasi:\begin{itemize}
    \item nella prima abbiamo analizzato il codice di LBC per trovare il punto  in cui iniettare gli attacchi scelti
    \item nella seconda abbiamo fatto guidare LBC su dei percorsi prestabiliti e raccolto i risultati
\end{itemize} 

La seconda fase si è divisa in più parti:\begin{itemize}
    \item nella prima LBC ha guidato sui percorsi senza nessun attacco iniettato
    \item nelle parti successive, abbiamo rieseguito le run, ciascuna volta con un diverso attacco iniettato.
\end{itemize}

I risultati raccolti hanno confermato la vulnerabilità dei CNN agli adversarial attacks. Nelle run con gli attacchi iniettati notato un forte calo nella stabilità della guida e un aumento delle collisioni.
Alcuni attacchi si sono rilevati più efficaci di altri, ma tutti hanno causato il fallimento di alcuni dei percorsi.\\

Il lavoro svolto si presta a svariati sviluppi e approfondimenti, due  dei quali sono:\begin{itemize}
    \item lo studio e lo sviluppo di difese dagli adversarial attacks. La libreria ART fornisce anche l'implementazione di metodi per rendere più robuste le CNN.
    Queste difese potrebbero essere inserite all'interno di un modello e valutate per la loro efficacia.
    \item studio e sviluppo di rilevatori di input modificati. Gli adversarial attacks modificano gli input in modo da essere impercettibili ad occhio umano. Una 
    soluzione a questo problema potrebbe essere la creazione di un rilevatore che invece riesca a individuare eventuali modifiche. Anche in questo caso
    la libreria ART  fornisce strumenti adatti per lo scopo.
\end{itemize}