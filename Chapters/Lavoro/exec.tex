\section{Iniezione degli attacchi}
LearningByCheating fornisce un modulo (run\_benchmark) grazie al quale valutare un agente sensorimotore (benchmark\_agent) su una serie di percorsi prestabiliti.
In questi percorsi l'agente deve arrivare a un punto della mappa preciso entro un tempo limite. Per ciascun percorso viene tenuto traccia dei semafori rossi ignorati, delle invasioni di corsia e delle collisioni. Inoltre viene registrato un video che mostra il veicolo mentre percorre il tragitto.
Nel video vengono mostrati anche gli waypoints generati dalla rete neurale in ogni momento. Nel Listing \ref{run} viene definito il loop principale di un percorso.
Ad ogni intervallo di simulazione ( $env.tick()$ aspetta l'esecuzione di un'azione),  l'agente riceve le osservazioni dai sensori ($env.get\_observations()$), le passa alla 
rete neurale che decide l'azione ($agent.run\_step(observations)$) e applica l'azione restituita attraverso i controllori ($env.apply\_control(control)$). Il percorso termina 
se il veicolo arriva a destinazione, scade il tempo, oppure avviene una collisione. L'ultima condizione non era presente nel codice originale ma è stata aggiunta per risparmiare tempo sulla singola run nel caso in cui avvenga un incidente. Per essere efficaci gli attacchi devono modificare gli input 
\textbf{prima} che arrivino alla rete, quindi la riga $agent.run\_step(observations)$ è stata individuata come il punto giusto nel quale iniettare gli attacchi prescelti.

Il metodo $run\_step$ è definito nella classe ImageAgent del  modulo $image$(\ref{img}).

La perturbazione dell'immagine avviene in:
\begin{lstlisting}[language=Python]
    _rgb= self.attack.generate(x=_rgb.cpu())
    _rgb = torch.FloatTensor(_rgb)
\end{lstlisting}
Il metodo $generate$ attua la modifica vera e propria dell'immagine. Il campo $self.attack$ rappresenta l'attacco iniettato. L'iniezione avviene nelle due righe aggiunte al costruttore:
\begin{lstlisting}[language=Python]
    self.adv = load_model('/home/piazzesi/Desktop/carla_lbc/ckpts/image')
    self.attack = load_attack(self.adv, 'hopskipjump')
\end{lstlisting}
Le funzioni $load\_model$ e $load\_attack$ sono state implementate nel modulo $attack$ (Listing \ref{att}). Si nota come la definizione  degli attacchi corrisponde semplicemente a una chiamata
di un costruttore. L'implementazione vera e propria  è fornita direttamente dai moduli dell'ART.

Le ultime modifiche sono state attuate nel metodo $forward$ della classe $ImagePolicyModelSS$ (Listing\ref{for}). Questa classe è l'implementazione della rete neurale e in $forward$ avviene il processo decisionale.
$forward$ viene anche usato nel metodo $generate$ per realizzare la perturbazione. Il problema è che $generate$ prende come unico parametro l'immagine RGB mentre $forward$ usa anche velocità e comando.
La soluzione scelta è stata la seguente: la velocità e il comando non vengono passati a $forward$ attraverso i  parametri di input, ma vengono recuperati  dalle variabili globali $\_speed$ e $\_command$ definite in $run\_step$. Usare variabili globali
non è sicuramente una pratica di buona programmazione, ma ha permesso l'esecuzione degli attacchi senza problemi. Infine , sempre per garantire la compatibilità tra ART  e modello è stata aggiunta la riga:
\begin{lstlisting}[language=Python]
    location_pred = torch.reshape(location_pred,(1,10))
\end{lstlisting}
La modifica è necessaria perchè il tensore location\_pred è della forma [2, 5] (bidimensionale con 5 elementi per dimensione). L'ART non supporta output multidimensionali, dunque il tensore viene convertito nella forma [1, 10].
\lstinputlisting[language=Python,caption=loop base di una run,label=run]{code/run_benchmark.py}
\lstinputlisting[language=Python, caption=ImageAgent,label=img]{code/image.py}
\lstinputlisting[language=Python,caption=attack.py,label=att]{code/attack.py}
\lstinputlisting[language=Python,caption=ImagePolicyModelSS,label=for]{code/image_model.py}