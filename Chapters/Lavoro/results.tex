\section{Risultati}
Per confrontare gli attacchi è stato eseguito $run\_benchmark$ sulla suite di percorsi $regular$.
Questa suite è composta da 4 mappe con determinate caratteristiche:\begin{itemize}
    \item NoCrashTown01-v3\begin{itemize}
        \item numero di veicoli:20
        \item numero di pedoni:50
    \end{itemize}
    \item NoCrashTown02-v3 \begin{itemize}
        \item numero di veicoli:15
        \item numero di pedoni:50
    \end{itemize}
    \item NoCrashTown01-v4 \begin{itemize}
        \item numero di veicoli:20
        \item numero di pedoni:50
    \end{itemize}
    \item NoCrashTown02-v4 \begin{itemize}
        \item numero di veicoli:15
        \item numero di pedoni:50
    \end{itemize}
\end{itemize}
Ciascuna mappa prevede l'esecuzione di tre percorsi, ognuno con  condizioni ambientali diverse. 
La valutazione degli attacchi si è svolta in due fasi:\begin{itemize}
    \item nella prima, i percorsi sono stati seguiti da un agente "puro", senza nessun attacco iniettato
    \item nella seconda fase, si eseguono le stesse run della prima fase, ma ogni volta con un diverso attacco iniettato.
\end{itemize}
I risultati prodotti dall'iniezione di ciascun attacco sono stati confrontati con quelli prodotti dalla run pura, in modo da valutarne l'efficacia.
Il confronto  è stato fatto sia in termini di percorsi portati a termine, sia andando ad analizzare e commentare i video prodotti.
\subsection{Run Pura}

\subsection{Iniezione di HopSkipJump}

\subsection{Iniezione di Spatial Transformation}

\subsection{Iniezione di Basic Iterative Method}

\subsection{Iniezione di NewtonFool}